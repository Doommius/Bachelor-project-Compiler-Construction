%Template by Mark Jervelund - 2015 - mjerv15@student.sdu.dk

\documentclass[a4paper,10pt,titlepage]{report}

\usepackage[utf8]{inputenc}
\usepackage[T1]{fontenc}
\usepackage[english]{babel}
\usepackage{amssymb}
\usepackage{amsmath}
\usepackage{amsthm}
\usepackage{graphicx}
\usepackage{fancyhdr}
\usepackage{lastpage}
\usepackage{listings}
\usepackage{algorithm}
\usepackage{algpseudocode}
\usepackage[document]{ragged2e}
\usepackage[margin=1in]{geometry}
\usepackage{enumitem}
\usepackage{color}
\usepackage{datenumber}
\usepackage{venndiagram}
\usepackage{chngcntr}
\usepackage{mathtools}
\usepackage{booktabs}
\DeclarePairedDelimiter{\ceil}{\lceil}{\rceil}

%lstlisting ting:
\definecolor{dkgreen}{rgb}{0,0.45,0}
\definecolor{gray}{rgb}{0.5,0.5,0.5}
\definecolor{mauve}{rgb}{0.30,0,0.30}
\lstset{frame=tb,
  language=C,
  aboveskip=3mm,
  belowskip=3mm,
  showstringspaces=false,
  columns=flexible,
  basicstyle={\small\ttfamily},
  numbers=left,
  numberstyle=\footnotesize,
  keywordstyle=\color{dkgreen}\bfseries,
  commentstyle=\color{dkgreen},
  stringstyle=\color{mauve},
  frame=single,
  breaklines=true,
  breakatwhitespace=false
  tabsize=1
}
\renewcommand{\lstlistingname}{Code} 

\setdatetoday
\addtocounter{datenumber}{0} %date for dilierry standard is today
\setdatebynumber{\thedatenumber}
\date{}
\setcounter{secnumdepth}{0}
\pagestyle{fancy}
\fancyhf{}
\title{dm546 - Compiler - symboltable - handin1}

\newcommand{\Z}{\mathbb{Z}}
\lhead{Compiler (DM546)}
\rhead{Mjerv15, Trpet15  \& Mojae15}
\rfoot{Page  \thepage \, of \pageref{LastPage}}
\counterwithin*{equation}{section}

\begin{document}
\begin{titlepage}
\centering
    \vspace*{9\baselineskip}
    \huge
    \bfseries
    1. Assignment \\
    \normalfont 
    Mark Jervelund, Troels Blicher Petersen \& Morten Jæger  \\
    (Mjerv15, Trpet15, Mojae15) \\
	\huge    
    Compiler (DM546)  \\[4\baselineskip]
    \normalfont
	\includegraphics[scale=1]{SDU_logo}
    \vfill\ 
    \vspace{5mm}
    IMADA \\

    \textbf{\datedate}  \bf{at 10} \\[2\baselineskip]
\end{titlepage}

\renewcommand{\thepage}{\roman{page}}% Roman numerals for page counter
\tableofcontents
\newpage
\setcounter{page}{1}
\renewcommand{\thepage}{\arabic{page}}


\newpage


\newpage
\section{Appendix}
\subsection{main.c}
\begin{lstlisting}
//
// Created by jervelund on 2/9/18.
//
#include "symbol.h"
#include <string.h>
#include <stdio.h>
#include <struct.h>
#include <stdlib.h>
#include <ctype.h>

int main(int argc, char **argv) {

//    char str[100]; // make sure that this size is enough to hold the single line
    int no_line = 1;

    SymbolTable *table = initSymbolTable();
    char buf[100];
    do {
        fgets(buf, sizeof buf, stdin);
        if (buf[strlen(buf) - 1] == '\n') {
            char *str;
            str = malloc((strlen(buf)-1));
            strcpy(str,buf);
            str[strlen(buf)-1] = 0;
            SYMBOL *symbol = putSymbol(table, str, Hash(str));
//            printSymbol(symbol);
        } else {
        break;
        }
//        dumpSymbolTable(table);
    } while (buf != "\0");

    dumpSymbolTable(table);

    return 1;


}
\end{lstlisting}
\subsection{symbol.c}
\begin{lstlisting}
#include <stddef.h>
#include <stdio.h>
#include <string.h>
#include "symbol.h"
#include "memory.h"


/*
 * Computes the hash function as seen below.
 *
 */
int Hash(char *str) {
    unsigned int length;
    length = (unsigned) strlen(str);
//    printf("the lenght of the string is %i \n",length);

    int k = (int) str[0];
    int i;
    int pointer = 1;

    while (pointer < length) {
//        printf("shifting k from %i ", k);
        k = k << 1;
//        printf("to %i \n",k);
        i = (int) str[pointer];
//        printf("loaded value i %i \n",i);
//        printf("adding i (%i) and k (%i) \n", i,k);
        k = i + k;
//        printf("sum is %i \n", k);

        pointer++;
    }
    return (k % HashSize);
}

/*
initSymbolTable returns a pointer to a new initialized hash table (of type
        SymbolTable)
*/
SymbolTable *initSymbolTable() {

    int i = 0;
    SymbolTable *table = Malloc(sizeof(SYMBOL) * HashSize);
    table->next = NULL;
    while (i < HashSize) {
        table->table[i] = NULL;
        i++;
    }
    return table;
}

/*
 *
 * scopeSymbolTable takes a pointer to a hash table t as argument and returns
 * a new hash table with a pointer to t in its next field.
 */
SymbolTable *scopeSymbolTable(SymbolTable *t) {
    SymbolTable *newTable = initSymbolTable();
    newTable->next = t;
    return newTable;
}


/*
 * putSymbol takes a hash table and a string, name, as arguments and inserts name into the hash table together with the associated value value. A pointer
 * to the SYMBOL value which stores name is returned.
*/
SYMBOL *putSymbol(SymbolTable *t, char *name, int value) {
    if (t == NULL) {
        return NULL;
    }
    SYMBOL *localCheck = checkLocal(t, name);
    //Symbol already exists
    if (localCheck != NULL) {
        return localCheck;
    } else {

        SYMBOL *symbol = Malloc(sizeof(SYMBOL));
        symbol->name = name;
        symbol->value = value;
        symbol->next = Malloc(sizeof(SYMBOL));

        //Placed in front of the list
        symbol->next = t->table[value];
        t->table[value] = symbol;
        return symbol;

    }
}


/*
    getSymbol takes a hash table and a string name as arguments and searches for
    name in the following manner: First search for name in the hash table which
    is one of the arguments of the function call. If name is not there, continue the
    search in the next hash table. This process is repeatedly recursively. If name has
    not been found after the root of the tree (see Fig. 1) has been checked, the result
    NULL is returned. If name is found, return a pointer to the SYMBOL value in
    which name is stored
    */
SYMBOL *getSymbol(SymbolTable *t, char *name) {
//    First check if t is null
    if (t == NULL) {
        return NULL;
    }

    SYMBOL *localCheck = checkLocal(t, name);

    //Symbol in local table?
    if (localCheck != NULL) {
        return localCheck;
    }

    if (t->next != NULL) {
        getSymbol(t->next, name);
    }

    //Symbol does not exists, should be an error message or such, later in the project
    return NULL;
}


/*
 * dumpSymbolTable takes a pointer to a hash table t as argument and prints all
 * the (name, value) pairs that are found in the hash tables from t up to the root.
 * Hash tables are printed one at a time. The printing should be formatted in a nice
 * way and is intended to be used for debugging (of other parts of the compiler).
*/
void dumpSymbolTable(SymbolTable *t) {
    if (t == NULL) {
        return;
    }

    printf("Printing symbol table:\n\n");

    for (int i = 0; i < HashSize; i++) {
        if (t->table[i] != NULL) {
            printSymbol(t->table[i]);
            printf("\n");
        }
//        SYMBOL *symbol = t->table[i];
//
//        if (symbol != NULL){
//            printf("Hash: \t%i ", i);
//            while (symbol != NULL){
//                printSymbol(symbol);
//                printf(" ");
//                symbol = symbol->next;
//            }
//            printf("\n");
//        }
    }
    printf("\n");

    dumpSymbolTable(t->next);

}

/*
 * Check the current table we are in for a value
 */
SYMBOL *checkLocal(SymbolTable *t, char *name) {
    int hashValue = Hash(name);


    SYMBOL *symbol = t->table[hashValue];
    if (symbol == NULL) {
        return NULL;
    } else {
        while (symbol != NULL) {
            if (strcmp(name, symbol->name) == 0) {
                return symbol;
            }
            symbol = symbol->next;
        }
    }

    //Hash value for the symbol exists, but the symbol is not in the table
    return NULL;
}

void printSymbol(SYMBOL *symbol) {
    printf("(%s, %i)", symbol->name, symbol->value);
}
\end{lstlisting}
\subsection{tests.c}
\begin{lstlisting}
// First file!
#include <stdio.h>
#include "symbol.h"
#include <string.h>

#define ANSI_COLOR_RED     "\x1b[31m"
#define ANSI_COLOR_GREEN   "\x1b[32m"
#define ANSI_COLOR_YELLOW  "\x1b[33m"
#define ANSI_COLOR_BLUE    "\x1b[34m"
#define ANSI_COLOR_MAGENTA "\x1b[35m"
#define ANSI_COLOR_CYAN    "\x1b[36m"
#define ANSI_COLOR_RESET   "\x1b[0m"
#define testString         "kitty"
#define testString2        "tesu"
#define testString3        "tets"

int main(int argc, char **argv) {
    int returnvalue = 1;
    int testpassed = 0;
    int totaltests = 0;



    printf("Testing hash function\n");
    totaltests++;
    int i = (Hash(testString));
    if (i == 199) {
        printf(ANSI_COLOR_GREEN     "Test %i - Hash function test PASSED\n"     ANSI_COLOR_RESET "\n", totaltests);
        testpassed++;
    } else {
        printf(ANSI_COLOR_RED     "Test %i - Hash functon test FAILED\n"     ANSI_COLOR_RESET "\n", totaltests);
        returnvalue = -1;
    }
    printf("\n");


    printf("Testing init table function\n");
    totaltests++;
    SymbolTable *table = initSymbolTable();
    if (table != NULL) {
        printf(ANSI_COLOR_GREEN     "Test %i - Table constructed PASSED\n"     ANSI_COLOR_RESET "\n", totaltests);
        testpassed++;
    } else {
        printf(ANSI_COLOR_RED     "Test %i - Table construction FAILED\n"     ANSI_COLOR_RESET "\n", totaltests);
        returnvalue = -1;
    }

    printf("\n");
    printf("Testing putsymbol function\n");
    totaltests++;
    SYMBOL *symbol = putSymbol(table, testString, Hash(testString));
    if (symbol != NULL) {
        if (symbol->value == Hash(testString) && (strcmp(symbol->name, testString) == 0)) {
            printf(ANSI_COLOR_GREEN     "Test %i - Symbol correctly made.\n"     ANSI_COLOR_RESET "\n", totaltests);
            testpassed++;
        } else {
            printf(ANSI_COLOR_RED     "Test %i - Symbol creation FAILED\n"     ANSI_COLOR_RESET "\n", totaltests);
            printf("Symbol was made but data was wrong");
            returnvalue = -1;
        }
    } else {
        printf(ANSI_COLOR_RED     "Test %i - Symbol creation FAILED\n"     ANSI_COLOR_RESET "\n", totaltests);
        returnvalue = -1;
    }

    printf("\n");
    printf("Testing getSymbol function\n");
    totaltests++;
    SYMBOL *symbol2 = getSymbol(table, testString);
    if (symbol2 != NULL) {
        if (symbol2->value == Hash(testString) && (strcmp(symbol2->name, testString) == 0)) {
            printf(ANSI_COLOR_GREEN     "Test %i - Symbol correctly returned.\n"     ANSI_COLOR_RESET "\n", totaltests);
            testpassed++;
        }
    } else {
        printf(ANSI_COLOR_RED     "Test %i - Symbol return FAILED\n"     ANSI_COLOR_RESET "\n", totaltests);
        returnvalue = -1;
    }

    printf("\n");
    printf("Testing putsymbol function again\n");
    totaltests++;
    symbol = putSymbol(table, testString, Hash(testString));
    if (symbol != NULL) {
        if (symbol->value == Hash(testString) && (strcmp(symbol->name, testString) == 0)) {
            printf(ANSI_COLOR_GREEN     "Test %i - Symbol correctly made.\n"     ANSI_COLOR_RESET "\n", totaltests);
            testpassed++;
        } else {
            printf(ANSI_COLOR_RED     "Test %i - Symbol creation FAILED\n"     ANSI_COLOR_RESET "\n", totaltests);
            printf("Symbol was made but data was wrong");
            returnvalue = -1;
        }
    } else {
        printf(ANSI_COLOR_RED     "Test %i - Symbol creation FAILED\n"     ANSI_COLOR_RESET "\n", totaltests);
        returnvalue = -1;
    }

    printf("\n");
    printf("Testing searching for a symbol, that is not in the table, but the hash for the symbol exists.\n");
    totaltests++;
    SYMBOL *symbol3 = putSymbol(table, testString2, Hash(testString2));

    symbol = getSymbol(table, testString3);

    if (symbol3 != NULL) {
        if (symbol == NULL) {
            printf(ANSI_COLOR_GREEN     "Test %i - NULL successfully received.\n"     ANSI_COLOR_RESET "\n", totaltests);
            testpassed++;
        }
    } else {
        printf(ANSI_COLOR_RED     "Test %i - Symbol search failed\n"     ANSI_COLOR_RESET "\n", totaltests);
        returnvalue = -1;
    }


    printf("Testing putSymbol function with a symbol which hash already is in the table\n");
    totaltests++;
    SYMBOL *symbol4 = putSymbol(table, testString3, Hash(testString3));

    if (symbol3 != NULL && symbol4 != NULL) {
        if (symbol3->value == Hash(testString2) && (strcmp(symbol3->name, testString2) == 0)) {
            if (symbol4->value == Hash(testString3) && (strcmp(symbol4->name, testString3) == 0)) {
                printf(ANSI_COLOR_GREEN     "Test %i - Symbol correctly made.\n"     ANSI_COLOR_RESET "\n", totaltests);
                testpassed++;
            }
        } else {
            printf(ANSI_COLOR_RED     "Test %i - Symbol creation FAILED\n"     ANSI_COLOR_RESET "\n", totaltests);
            printf("Symbol was made but data was wrong");
            returnvalue = -1;
        }
    } else {
        printf(ANSI_COLOR_RED     "Test %i - Symbol creation FAILED\n"     ANSI_COLOR_RESET "\n", totaltests);
        returnvalue = -1;
    }


    printf("Testing scopeSymbolTable function\n");
    totaltests++;
    SymbolTable *newTable = scopeSymbolTable(table);
    if (newTable != NULL) {
        printf(ANSI_COLOR_GREEN     "Test %i - Table constructed PASSED\n"     ANSI_COLOR_RESET "\n", totaltests);
        testpassed++;
    } else {
        printf(ANSI_COLOR_RED     "Test %i - Table construction FAILED\n"     ANSI_COLOR_RESET "\n", totaltests);
        returnvalue = -1;
    }

    printf("\n");
    printf("Testing putsymbol function in new table\n");
    totaltests++;
    SYMBOL *symbol5 = putSymbol(newTable, testString, Hash(testString));
    if (symbol5 != NULL) {
        if (symbol5->value == Hash(testString) && (strcmp(symbol5->name, testString) == 0)) {
            printf(ANSI_COLOR_GREEN     "Test %i - Symbol correctly made.\n"     ANSI_COLOR_RESET "\n", totaltests);
            testpassed++;
        } else {
            printf(ANSI_COLOR_RED     "Test %i - Symbol creation FAILED\n"     ANSI_COLOR_RESET "\n", totaltests);
            printf("Symbol was made but data was wrong");
            returnvalue = -1;
        }
    } else {
        printf(ANSI_COLOR_RED     "Test %i - Symbol creation FAILED\n"     ANSI_COLOR_RESET "\n", totaltests);
        returnvalue = -1;
    }


    printf("\n \n");

    printf("Tests PASSED %i\n", testpassed);
    printf("\n \n");


    if (totaltests - testpassed != 0) {
        printf(ANSI_COLOR_RED     "test FAILED %i\n"     ANSI_COLOR_RESET "\n", totaltests - testpassed);
    }

    dumpSymbolTable(newTable);

    return returnvalue;

}
\end{lstlisting}
\newpage
{%
\centering
    \huge
    \bfseries
    \vspace{5mm}
CC, Spring 2018\\
Exam project, part 1\\
\vspace{5mm}
\begin{tabular}{|l|l|}
\hline
Group & 1 \\ \hline
\end{tabular}
\\
\vspace{10mm}
\begin{tabular}{@{}ll@{}}
\toprule
\multicolumn{1}{|l|}{Name}      & \multicolumn{1}{l|}{Mark Wolff Jervelund } \\ \midrule
\multicolumn{1}{|l|}{Birthday}  & \multicolumn{1}{l|}{280795} \\ \midrule
\multicolumn{1}{|l|}{Login}     & \multicolumn{1}{l|}{mjerv15@student.sdu.dk} \\ \midrule
\multicolumn{1}{|l|}{Signature} & \multicolumn{1}{l|}{\includegraphics[scale=0.3]{mark_sign}} \\ \midrule
                                &                       \\ \midrule
\multicolumn{1}{|l|}{Name}      &  \multicolumn{1}{l|}{Troels Blicher Petersen} \\ \midrule
\multicolumn{1}{|l|}{Birthday}      &   \multicolumn{1}{l|}{230896} \\ \midrule
\multicolumn{1}{|l|}{Login}      &  \multicolumn{1}{l|}{trpet15@student.sdu.dk} \\ \midrule
\multicolumn{1}{|l|}{Signature}      & \multicolumn{1}{l|}{\includegraphics[scale=0.08]{troels_sign} } \\ \midrule
                                &                       \\ \midrule
\multicolumn{1}{|l|}{Name}     &  \multicolumn{1}{l|}{Morten Kristian Jæger} \\ \midrule
\multicolumn{1}{|l|}{Birthday}      &  \multicolumn{1}{l|}{030895} \\ \midrule
\multicolumn{1}{|l|}{Login}     &   \multicolumn{1}{l|}{mojae15@student.sdu.dk} \\ \midrule
\multicolumn{1}{|l|}{Signature}    &  \multicolumn{1}{l|}{\includegraphics[scale=0.3]{morten_sign}} \\ \midrule
\end{tabular}
\\
\vspace{10mm}
This report contains a total of \underline{ 15 } pages.


}
\end{document}
